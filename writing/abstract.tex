\loadgeometry{myAbstract}
\chapter*{Anotácia\markboth{Abstrakt}{Abstrakt}}
%\addcontentsline{toc}{chapter}{\protect\numberline{}Abstrakt}
\label{abstrakt}
\begin{center}
\textbf{Fakulta Informatiky a Informačných Technológií}\\
\textbf{Slovenská Technická Univerzita}\\
\end{center}
\begin{tabular}{ p{15em} p{15em} }
Meno: & Bc. Matúš Cimerman\\
Vedúci diplomovej práce: & Ing. Jakub Ševcech, PhD.\\
Diplomová práca: & Analýza prúdu prichádzajúcich udalostí použitím rôznych metód pre analýzu údajov\\
Študijný program: & Informačné systémy\\
Máj 2016
\end{tabular}

%here (slovak version)
V súčasnosti pozorujeme narastajúcu záujem a potrebu analyzovať dáta v čase ich vzniku. Spracovanie a analýza prúdov dát predstavuje komplexnú úlohu, pričom je dôležité poskytnúť riešenie s nízkou odozvou. Použitie a interpretácia analytických metód v doméne spracovania údajov uvádza známy problém. Jedným z problémov je použitie tradičných dávkových metód pre prúdy dát a je teda potrebné hľadať iné alternatívy. Ďalšou výzvou je interpretovanie výsledného modelu a výsledkov. V oblasti prúdov dát je avšak tomuto fenoménu venovaná len nepatrná pozornosť. Sústreďujeme sa najmä na problém interpretácie výsledkov doménovému expertovi, pričom predpokladáme, že doménový expert nepotrebuje mať detailné znalosti o fungovaní modelu.\\
Navrhujeme použitie rozhodovacieho stromu v kontexte klasifikácie prúdu dát, ktorý používa Hoeffdingovu mieru pre výber najlepšieho rozhodovacieho atribútu so stanovenou istotou. Pre sa vysporiadanie so zmenami aplikujeme algoritmus ADWIN, ktorý adaptívne detekuje zmeny v prúde dát. Zameriavame sa pritom na jednoduchosť vybranej metódy a interpretovateľnosť výsledkov. Pre doménových expertov je nevyhnutné aby boli tieto požiadavky splnené, pretože nebudú potrebovať detailné znalosti z domén ako strojové učenie alebo štatistika. Avšak, znalosti dát, ich kontextu a významu jednotlivých atribútov sú nevyhnutné. Naše riešenie overujeme implementovaním vizualizácie a overením kvalitatívnym experimentom.


%=== ENGLISH VERSION===% 
\emptydoublepage
\chapter*{Annotation\markboth{Abstract}{Abstract}}
%\addcontentsline{toc}{chapter}{\protect\numberline{}Abstract}
\label{abstract}
\begin{center}
\textbf{Faculty of Informatics and Information Technologies}\\
\textbf{Slovak University of Technology}\\
\end{center}
\begin{tabular}{ p{10em} p{15em} }
Name: & Bc. Matúš Cimerman\\
Supervisor: & Ing. Jakub Ševcech, PhD.\\
Diploma thesis: & Stream analysis of incoming events using different data analysis methods\\
Course: & Information systems\\
2016, May
\end{tabular}

%here (english version)
%TODO: prelozit aktualny abstrakt do angl.
TODO: prelozit aktualny abstrakt

\emptydoublepage
\loadgeometry{myText}
