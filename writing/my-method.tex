\chapter{Klasifikácia rozhodovacími stromami}
\label{Klasifikácia rozhodovacími stromami}
Problém klasifikácie a jej definícia je podrobne opísaný v kapitole \ref{ulohy-klasifikacia}. V skratke, cieľom je nájsť funkciu $y = f(x)$, kde $y$ je skutočná trieda objektu/vzorky z prúdu dát a $x$ sú atribúty danej vzorky. Potom vieme pomocou funkcie $f(x)$ klasifikovať nové vzorky do triedy $y'$ s istou pravdepodobnosťou. My sa sústreďujeme na úlohu klasifikácie v doméne prúdu dát. V tejto kapitole opisujeme návrh spracovania prúdu dát, výber a aplikáciu metódy rozhodovacích stromov nad prúdom dát.

%%%%%%%%%%%%%%%%%%%%%%%%%
% Spracovanie prudu dat %
%%%%%%%%%%%%%%%%%%%%%%%%%
\section{Spracovanie prúdu dát}
\label{method-spracovanie-prudu-dat}

Spracovaniu prúdu dát venujeme samostatnú podkapitolu, pretože si zaslúži špeciálnu pozornosť a rozdielny prístup v porovnaní so spracovaním statickej kolekcii dát. Navrhovaná metóda je všeobecne použiteľná na problémy klasifikácie pre prúdy dát. Znamená to, že spracuje dáta v takmer reálnom čase, poskytne odpoveď a teda aj vytvorený model okamžite a je schopná adaptávacie na zmeny. Pre splnenie týchto požiadaviek je potrebné venovať samostatnú pozornosť spracovaniu prúdu dát, teda požadujeme aby navrhovaná metóda spĺňala nasledujúce kritéria \citep{cimerman2015prudy}:
\begin{itemize}
	\item \textit{Odolnosť voči chybám} z pohľadu architektúry spracujúcej dáta. Chybné alebo chýbajúce dáta môžu mať kritický dopad na správne fungovanie a kvalitu klasifikačného modelu.
	\item \textit{Spracovanie v reálnom čase} je opäť dôležité pre správne fugnovanie výsledného modelu, pretože model je aktualizovaný a prispôsobovaný zmenám v dátach kontiunálne. Oneskorenie niektorých správ, napríklad o 24 hodín čo je bežná prax pri ETL\footnote{ETL je proces, či architektonický vzor prenosu dát medzi viacerými častami databázových systémov  a aplikáciami, tento vzor je často používaný pre dátové sklady, skratka znamená Extrahuj, Transformuj a Načítaj (angl. Extract, Transform, Load)} procesoch, by mohlo mať nežiadúce následky vo forme skresleného modelu.
	\item \textit{Horizontálna škálovateľnosť} komponentu, ktorý spracuje prúd dát. Táto vlastnosť podporuje splnenie predchádzajúcich požiadaviek. Pod horizontálnou škálovateľnosťou chápeme to, že je možné zvýšiť výkonnosť celého systému pridaním fyzického uzla bez akýchkoľvek výpadkov. Táto požiadavka implikuje podmienku distribuovanej povahy riešenia.
\end{itemize}

S cieľom splniť tieto požiadavky navrhujeme použiť nasledujúce programovacie rámce a systémy:
\begin{itemize}
	\item \textit{Storm}\footnote{http://storm.apache.org/} je programovací rámec vytvorený pre spracovanie dát v reálnom čase. Storm poskytuje možnosti škálovateľnej architektúry, ktorá je naviac odolná voči chybám na úrovni kvality dát. Programovanie nad týmto rámcom je možné v každom programocom jazyku, ktorý je možné skompilovať do Java bajtkódu a vykonávať v JVM\footnote{Virtuálny stroj Java (angl. Java virtual machine)}. Storm poskytuje aplikovať akýkoľvek programovací vzor, model ktorý poskytuje je vyjadrený, resp. vytvára acyklický orientovaný graf zostrojený z tzv. prameňov a skrutiek.
	\item \textit{Kafka}\footnote{https://kafka.apache.org/} je distribuovaná platforma pre spracovanie prúdov dát. Kafka je vhodná na budovanie apliikácií, ktoré potrebujú spracovať zdroje dát v reálnom čase a vymieňať tieto dáta medzi aplikáciami. Poskytuje možnosť publikovat (angl. publish) a predplatiť (angl. subscribe) prúdy dát. Kafka je postavená na modely fronty správ, pričom si tieto správy udržiava v pamäti a na disk ich replikuje pre prípad zlyhania. Kafka poskytuje distribuované a paralelné spracovanie dát čo robí tento nástroj vhodný v aplikáciach reálneho sveta.
\end{itemize}

Nasledujúci obrázok schematicky popisuje architektúru spracovania prúdu dát potrebnú pre správne fungovanie metódy pre klasifikáciu prúdu dát s použitím rozhodovacích stromov.
\myFigure{images/architecture}{Architektúra potrebná pre klasifikáciu prúdu dát v takmer reálnom čase. Architektúra pozostáva z troch úrovní. V časti predspracovania dát sú dáta zbierané zo zdroja prúdu dát a transformované do potrebnej podoby vhodnej pre ďalší krok. V kroku učenia modelu pre klasifikáciu prúdu dát je semi-automaticky vybraný vhodný algoritmus a atribúty a vytvorený klasifikačný model. Posledný krok obsahuje webovú službu, ktorá poskytuje Web API pre dotazovanie modelu. V tomto kroku tiež prezentujeme výsledky modelu v podobe vizualizácie používateľovi. Kafka je použitá na prenos správ medzi jednotlivými časťami aplikácie, správy sú rozdelené do rôznych tém podľa typu správy.}{architecture}{0.45}{h!}\label{fig:architecture}
\par
Web API rozhranie implementujeme ako asynchrónny web server. Volania Web API servera sú popísané v tabuľke \ref{tab-web-api}. Hlavným cieľom Web API rozhrania je možnosť klienta požiadať o posielanie aktualizácií modelu. Tieto aktualizácie sú odovzdávané v jednosmernej prevádzke vo forme Server-Sent Events (SSE) kde klient počúva a server môže posielať aktualizácie. Pri tejto forme komunikácie nieje potrebné pri každej správe otvárať nové TCP spojenie čo je vhodné práve pre prúdové aplikácie. Pre každého nového klienta server vytvára nového Kafka konzumera. Nový konzumer je vždy vytváraný, pretože každý klient môže mať rôznu rýchlosť spracovania správ a teda nastavený iné posunutie (angl. offset) kafka konzumera.

\begin{table}[!htp]
\centering
\begin{tabular}{|r|l|}
\hline
\textbf{Cesta API volania} & \textbf{Popis} \\ \hline
/status/ & Vráti status web servera \\ \hline
/tree/ & Otvorí SSE spojenie a začne posielať aktualizácie \\ \hline
/node/\textit{node id}/ & Vráti informácie o danom uzle stromu \\ \hline
\end{tabular}
\caption{Popis Web API rozhrania webservera.}
\label{tab-web-api}
\end{table}

\section{Metóda klasifikácie prúdu dát}
\label{method-klasifikacia-prudu-dat}
Cieľom je klasifikácia prúdu dát, pričom vytvorený model je pripravený na použitie takmer okamžite po prečítaní prvých vzoriek dát. Model sa tiež prispôsobuje zmenám a do istej miery sezónnym efektom v dátach. Základ metódy pre klasifikáciu sme zvolili state-of-the-art algoritmus rozhodovacích stromov, ktorý používa Hoeffdingovu mieru \citep{domingos2000mining, gaber2005mining, krempl2014open}. Hoeffdingova miera je použitá na rozhodnutie, či bol prečítaný dostatočný počet vzoriek na to aby sa mohol uzol v strome zmeniť na rozhodovací uzol. Táto miera zabezpečuje to, že sa výsledný model asymptoticky blíži svojou kvalitou k tomu, ktorý by vznikol podobnou metódou pre statické dáta. Zároveň má táto miera vlastnosť, že  dôvera v presnosť modelu exponenciálne rastie s lineárnym nárastom počtu prečítaných vzoriek. Hoeffdingova miera je definovaná používateľom parametrom spoľahlivosti $\delta$, kde spoľahlivosť je $(1-\delta) \in <0,1>$. Metrika kvality vzorky $G$ môže byť použitá ľubovoľná, napríklad informačný zisk (angl. information gain).
\par
Metóda potrebuje na trénovanie označkované numerické alebo kategorické dáta do viacerých tried. Dáta musia byť vo forme n-tíc $(x_1, x_2, ..., x_n | y)$ kde $x$ sú atribúty vzorky a $y$ skutočná trieda vzorky. Nami vybraný potrebuje len minimum parametrov, ktoré je potrebné nastaviť pre správne fungovanie. Jedným z nich je minimálny počet spracovaných vzoriek pred vytvorením prvého modelu. Týmto je možné minimalizovať prvotnú nepresnosť počiatočného modelu. Výsledný model je jednoduchý na reprezentáciu vďaka možnosti jeho intuitívnej interpretácii rozhodovacím stromom. Rozhodovací strom pozostáva z rozdeľovacích uzlov (angl. split node), tiež niekedy nazývané testovacie uzly, a listov (angl. leaf). V rozhodovacích uzloch sa vykonáva testovanie vzorky a jej posunutie do jednej z nasledujúcich vetiev alebo listu stromu. Ak vzorka narazí na list znamená to, že bola klasifikovaná do istej triedy, ktorú opisuje daný list. Takto vytvorený model je použiteľný na klasifikáciu v rôznych aplikáciách.
\par
Problémom rozhodovacích stromov je najmä ich šírka. Klasifikátory, ktoré používajú modely a algoritmy rozhodovacích stromov môžu podľa dát byť priveľmi široké. Tento problém môže mať za následok preučenie (angl. overfitting) modelu, ktorý bude vedieť klasifikovať veľmi dobre trénovacie dáta, resp. dáta zo začiatku prúdu, ale na nových dátach bude veľmi nepresný. Tento problém nastáva najmä pri spojitých číselných atribútoch a ich nerovnomernej distribúcii. Existuje niekoľko známych spôsobov ako sa stýmto nežiadúcim javom vysporiadať, jedným z nich je pre-prerezávanie (angl. pre-pruning) stromu. Tento spôsob aplikujeme aj v našej metóde priadním nulového atribútu $X_0$ do každého uzla, ktorý spočíva v nerozdeľovaní daného uzla. Takže uzol sa stane rozhodovacím iba, ak je metrika $G$, so spoľahlivosťou $1-\delta$, lepšia ako keby sa uzol nezmenil na rozhodovací.
\par
Metóda sa musí vysporiadať so zmenami v dátach, pretože zmeny v dátach sú prítomné v takmer všetkých prúdoch reálneho sveta. Zmeny môžu mať rôzny charakter, napríklad náhly kedy zmena nastane nečakane alebo postupný kedy sa zmena deje dlhú dobu a pomaly. Výsledný model musí pre udržanie svojej presnosti zohladniť tieto zmeny. Metóda používa algoritmus \textit{ADWIN} z anglického Adaptive Windowing \citep{Hutchison2009}. Tento algoritmus nepožaduje žiadne nastavenia parametrov používateľom ako napríklad veľkosť posuvného okna. Jediným parametrom je hodnota istoty $\delta$ s akou bude algoritmus detekovať zmeny v prúde dát. Myšlienka ADWIN spočíva v tom, že nenenastala žiadna zmena v priemernej hodnote vybranej metriky v okne. Ak je detekovaná zmena, v uzle začne narastať alternujúci podstrom. Tento podstrom musí spracovať definovaný minimálny počet vzoriek. Potom, ak je kvalita podstromu vyššia ako kvalita podstromu, z ktorého začal narastať, starý podstrom je nahradený alternujúcim podstromom. Naraz môže existovať niekoľko alternujúcich podstromov, pričom môže nastať situácia kedy ani jeden z nich nebude mať vyššiu kvalitu a nesplní Hoeffdingovu mieru preto aby nahradil starý podstrom. Výsledok tohto algoritmu chceme detailne prezentovať používateľovi vo forme vizualizácie, ktorá je detailnejšie popísaná v nasledujúcej podkapitole. 
\par
Táto metóda potrebuje pamäť úmernú $O(ndvc)$ kde $n$ je počet uzlov stromu spolu s alternatívnymi stromami, $d$ je počet atribútov, $v$ je maximálny počet hodnôt na atribúť a $c$ je počet tried. Znamená to teda, že pamäťová náročnosť algoritmu je závislá od štatistík, ktoré si strom udržiava, a úplne nezávislá od počtu spracovaných vzoriek z prúdu dát. Časová zložitosť spracovania jednej vzorky je $O(ldcv * log(w))$ kde $l$ je maximálna hĺbka stromu a $w$ je šírka ADWIN okna. Nakoľko algoritmus ADWIN používa variant techniky exponenciálnych histogramov s cieľom kompresie okna $w$, nemusí si celé okno explicitne udržiavať \citep{datar2002maintaining}. Vďaka tomu je časová a pamäťová náročnosť prechodu cez takéto okno o veľkosti $w$ len $O(log(w))$.
\par
Metódu implementujeme ako rozšírenie nad API, ktoré poskytuje nástroj Massive Online Analysis (MOA). Diagram tried implementácie je popísaný v prílohe XX.%TODO priloha moa impl
 Metódu by bolo možné jednoducho počítať distribuovane napríklad pomocu použitím rámca Storm.

%%%%%%%%%%%%%%%%%%%%%%%%%%%%%%%%%%%%%%%
% Prezentacia vysledkov pouzivatelovi %
%%%%%%%%%%%%%%%%%%%%%%%%%%%%%%%%%%%%%%%

\chapter{Vizualizácia modelu rozhodovacích stromov}
\label{my-method-prezentacia-vysledkov}
V situácii keď potrebuje doménovy expert vytvoriť klasifikačný model s použitím dát reálneho sveta, je často potrebná najprv detailná znalosť dát, ktorú doménový expert, predpokladáme má.  Následne preto, aby vedel vytvoriť správny model potrebuje mať detailné znalosti o fungovaní klasifikačných metód a algoritmov. Cieľom našej metódy je odbremeniť experta od nutnosti mať detailné znalosti o fungovaní modelu a algoritmov. Zameriavame sa teda na prezentáciu dôležitých informácií, ktoré potrebuje pre správne pochopenie modelu a následné rozhodnutia. Výber atribútov a algoritmov, ktoré budú použité na trénovanie modelu je bez nutnosti interakcie používateľa v zmysle nastavovania parametrov a výberu metódy. 
\par
Keďže náš predpoklad je, že používateľ nemusí mať detailné znalosti o fungovaní algoritmov a klasifikačných metód, je dôležité vysvetlenie výsledného modelu. Znamená to, že je dôležité aby pre používateľa nebol vzniknutý model len čierna skrinka (angl. black-box), ktorá s nejakou úspešnosťou dokáže klasifikovať prúdy dát. Navrhujeme preto vizualizáciu výsledného modelu. Vizualizácia je vo forme rozhodovacieho stromu, ktorý je jednoduchý na pochopenie aj bez predchádzajúcich znalostí o rozhodovacích stromov \citep{nguyen2015survey}. Sústreďujeme sa najmä na zobrazenie zmien (angl. concept drift) v dátach, ktoré sa odzrkadlia zmenou modelu.
\par
Hlavným prípadom použitia používateľom - doménovým expertom je nasledovný:
\begin{enumerate}
	\item Vytvorenie modelu rozhodovacieho stromu.
	\item Používateľ používa vytvorený model na strategické rozhodovanie v podniku.
	\item Používateľ siahne po vizualizácií, ak sa výrazne zníži kvalita modelu alebo potrebuje použévateľ lepšie pochopiť čo viedlo k vzniku aktuálneho modelu.
	\item Vizualizáciu je možné pozastaviť, pretože model sa inak neustále meni vyvíja v závislosti od frekvencie a distribúcie prúdiacich dát.
\end{enumerate}

\section{Návrh vizualizácie}
\label{navrh-vizualizacie}
Vizualizáciu navrhujeme implementovať ako webovú aplikáciu. Zameriavane sa pritom na to, aby vizualizácia bola online a teda zobrazovala vždy aktuálny stav modelu. Zobrazenie online meniacej sa vizualizácie modelu je pri modeloch prúdu dát dôležité pre lepšie pochopenie modelu. Táto vizualizácia má slúžiť najmä na zobrazenie priebehu učenia modelu v ktoromkoľvek momente s dôrazom na zobrazenie zmien, ktoré model ovpyvnili. V našej práci sme sa rozhodli vizualizovať model rozhodovacích stromov. Tieto rozhodovacie stromy, známy aj ako Hoeffdingove stromy, používajú Hoeffdingovu mieru ako mieru istoty pre výber ideálneho atribútu rozdeľovacieho uzla. Táto metrika nám napovedá to, či a ako sa model zmení. Vo vizualizácií využívame túto metriku na zobrazenie kvality listov alebo rozhodovacích uzlov stromu. Pomocou animácie zobrazujeme zmeny modelu, takže to prirodzene predstavuje evolúciu stromu. Na obrázku \ref{fig:vis-navrh} je zobrazený záber z vizualizácie.
\myFigure{images/6_navrh}{Vizualizácia dvoch iterácií modelu rozhodovacieho stromu. Modrá šípka zobrazuje prechod z pôvodného stavu do nového. Každý nový uzol v tejto vizualizácií vznikol samostatne a bol zobrazený animáciou. Model na obrázku bol vytvorený na syntetickom datasete.}{vis-navrh}{0.34}{h!}\label{fig:vis-navrh}
\par
Hodnota Hoeffdingovej miery je použitá na zafarbenie listov a rozhodovacích uzlov stromu. Listy stromu sú zafarbované na škále bielej a čiernej farby. Čím je nižšia Hoeffdingova miera v danom liste tým je farba listu tmavšia. V ideálnom prípade sa teda každý čierny list zmení na rozdeľovací uzol. Niekedy môže nastať prípad kedy sa aj menej tmavý list rozdelí. Je to z dôvodu, že Hoeffdingova miera je porovnávaná voči rozdieľu chybovosti s rozostupom pozorovaní. Ak je tento rozostup nastavený napríklad na 200 vzoriek a práve v tomto krátkom okne sa Hoeffdingova miera zmení natoľko, že sa list rozdelí, potom sa nestihne prefarbiť na čierno. Tento jav sme pozorovali zriedka na reálnych dátach a aj na syntetických dátach. Pri rozdeľovaní listu je pôvodný list aj s vetvou vyfarbený na červeno a postupne vymizne. Na obrázku \ref{fig:vis-hb-color} je zobrazený prechod rozdeľovania listu na rozdeľovací uzol. V prípade, že v strome nenastala zmena strom sa neprekreslí a aktualizujú sa iba farby uzlov. Animácia, ktorá zobrazuje rozdeľovanie listu zámerne ponecháva zobrazený aj pôvodný list práve preto aby mohol používateľ dobre pozorovať zmenu, ktorá nastala.
\myFigure{images/6_hb-color}{Na obrázku je pre jednoduchosť zobrazený strom po spracovaní prvých vzoriek, preto na začiatku nie je vytvorený žiadny rozhodovací uzol. Šípka zobrazuje rozdeľenie listu na rozhodovací uzol.}{vis-hb-color}{0.34}{h!}\label{fig:vis-hb-color}
\par
Keď nastane v dátach zmena, ktorá ovplyvní model, začne v danom rozhodujúcom uzle narastať alternatívny podstrom. Tento alternatívny podstrom začne narastať v rozhodovacom uzle, ktorého kvalita sa v čase začala znižovať. Ako detektor zmeny používame algoritmus ADWIN, ktorý je implementovaný v nástroji MOA. Pre samotnú vizualizáciu je tento detektor skrytý. Vizualizácia, ktorá slúži len ako front-end pre používateľa je informovaná o zmene z prúdu dát modelu stromov. Uzol, z ktorého začal narastať alternujúci podstrom, je označený červenou farbou. Červená farba zobrazuje vysokú pravdepodobnosť, že bude podstrom od daného uzla vymenený za jeho alternujúci podstrom. Okrem červenej farby v rozhodovacích uzloch je zobrazené aj explicitné upozornenie, že nastala zmena. Na obrázku \ref{fig:vis-change} je zobrazený model stromu s nastávajúcou zmenou. Výmena pôvodného postromu za jeho alternatívny podstrom je zobrazená rovnakou animáciou ako pri rozdeľovaní listu na rozhodovací uzol.
\myFigure{images/6_change}{Vizualizácia kde existujú tri alternujúce podstromy (nie sú zobrazené). Upozornenie hovorí len o jednej zmene, pretože predchádzajúce dve zmeny, ktoré viedli k vzniku alternatívnych podstromov nastali skôr a nestihli zmeniť model stromu. Niektoré vetvy stromu nie sú úplne dokreslené kvôli lepšej čitateľnosti obrázku. Nečitateľnosť niektorých rozhodovacích pravidiel je v reálnej webovej aplikácií vyriešená posunutím pravidla do popredia po nájdením myšou na pravidlo.}{vis-change}{0.34}{h!}\label{fig:vis-change}

\section{Implementácia vizualizácie}
\label{Implementácia vizualizácie}
Vizualizácia bola implementovaná ako webová aplikácia pomocou knižnice D3.js\footnote{https://d3js.org/} v jazyku JavaScript. Knižnica D3.js umožňuje naviazať dáta na objektový model dokumentu (angl. Document Object Model, DOM) a jednoducho aplikovať dátovo riadené transformácie na dokument. D3 nie je monolitický programovací rámec vytvorený s cieľom poskytnúť každú funkciu, ktorú by mohol programátor potrebovať. Naopak, D3 rieši problém manipulácie a zmien DOM na základe dát. Pomocou tejto knižnice je jednoduché vytvoriť animované vizualizácie s rôznymi prechodmi, ktoré budú riadené dátovým modelom. Hlavným dôvodom prečo sme sa rozhodli pre túto knižnicu je možnosť funkcionálneho programovania, dátovo riadené vizualizácia a abstrakcia od vytvávarania SVG grafiky.
\par
Webová aplikácia získáva potrebné dáta vo formáte JSON z Web API. Toto web API je implementované pomocou, už spomenutej, architektúry Server-Sent Events (SSE). Z pohľadu vizualizácie to znamená, že prijíma prúd dát bez nutnosti otvárať TCP, resp. HTTP, spojenie vždy keď je potrebné prečítať novú vzorku. Znamená to, že server je schopný posielať aktualizácie webovej aplikácií bez toho aby o tieto aktualizácie musela žiadať samotná webová aplikácia, napríklad periodicky. Toto je veľkou výhodou, pretože otvárať spojenie pri každej novej vzorke predstavuje veľké nepotrebné zaťaženie naviac. Okrem toho, webové prehliadače efektívne implementujú tento štandard a bez problémov je možné prijímať prúdy, ktoré generujú dáta s vysokou frekvenciou. Ukážka otvorenia SSE a asynchrónne spracovanie správy:
\begin{lstlisting}[language=JavaScript]
let eventStream = new EventSource(API_STREAM);
eventStream.addEventListener('tree', function (response) {
    let responseData = JSON.parse(response.data);

        BFT(responseData, addTreeIncrement).then((complete)=>{
            if (_.isEqual(complete, "success")) {
                log.info('Tree successfully processed');
            }
        removeAndFadeOutOldNodes(responseData, root);
    });
});
\end{lstlisting}
Dôležitou časťou spracovania údajov z tohto prúdu dát je práve asynchrónne spracovanie. Je to preto, že vizualizácia nežiada o aktualizácie manuálne alebo periodicky z dôvodu šetrenia prostriedkov a zaťaženia výpočtového výkonu sieťovej prevádazky. Aktualizácie sú posielané vo vysokej frekvencii serverom. Keďže vizualizácia obsahuje animácie, pričom každá trva približne 500ms, nový model stromu je potrebné spracovať nezávisle od nových vzoriek z prúdu dát. Keby sme každú novú vzorku spracovali v čase jej príchodu vizualizácia by sa stala nečitateľná a nepoužiteľná. Po otvorení SSE spojenia a prijatí prvej aktualizácie vo forme modelu rozhodovacieho stromu vo formáte JSON je model vykreslený. Po prijatí ďalšej aktualizácie je model vykreslený, ak boli ukončené všeky animácie z predchádzajúceho vykresľovania. Vykreslovanie modelu stromu prebieha prehľadávaním do šírky kde každý uzol je vykreslený samostatne s vlastnou animáciou. Je to z dôvodu aby bola celý animácia plynulá. Algoritmus vykresľovania stromu je popísaný pseudokódom v algoritme \ref{alg:visualization}. Výpočtová zložitosť pridania uzlu je $O(|V| + |E|)$ kde $|V|$ je počet uzlov a $|E|$ je počet hrán. Skutočná časová zložitosť je znásobená časom každej animácie pri pridaní uzla, pri aktualizácií uzla je to len niekoľko milisekúnd pretože netreba prekresliť žiadnu časť SVG.
\par
Navrhnutá metóda pre vizualizáciu rozhodovacích stromov, ktoré sa učia online, poskytuje online animovanú vizualizáciu procesu tvorby modelu. Dôraz je kladený na zobrazenie zmeny modelu. Túto zmenu zvýrazňujeme červeným zafarbením rozhodovacích uzlov, ktorých sa zmena v dátach dotkla. Pre správne zobrazenie vizualizácie sme čelili viacerým technickým výzvam a to najmä správnemu vykresľovaniu modelu stromu pri prijímaní prúdu s vysokou frekvenciou aktualizácií. Pri prezeraní vizualizácie môžep používateľ zastaviť vizualizáciu alebo opäť spustiť. Okrem toho nie je možné inak ovládať vizualizáciu, pretože bola navrh Túto hypotézu overujeme kvalitatívnym experimentom s niekoľkými doménovými expertami.
\par
% ====== ALGORITMUS visualization ====== %
\begin{algorithm}[H]
	\label{alg:visualization}
	\SetAlgoLined
    \SetKwInOut{Input}{Input}
    \SetKwInOut{Output}{Output}
    \SetKwFunction{visualize}{visualize}
    \SetKwFunction{breadthFirstSearch}{breadthFirstSearch}
    \SetKwProg{vis}{function}{}{}
    \SetKwProg{bfs}{function}{}{}
    
    \bfs{\breadthFirstSearch{$tree data$}}{
    	\Input{parsed tree JSON data}
    	
    	\For{\textbf{each} $node$ \textbf{in} $tree data$} {
			\eIf{node doesn't exists}{
				\textit{await} add new node to tree
			}{
				\textit{await} update node information
			}
		}
    }

    \vis{\visualize{$stream$}}{
    	\Input{Opened SSE stream}
    	\Output{Decision tree model SVG in DOM}
    	
    	\While{message received}{
			treeData $\leftarrow$ parse message data JSON\;
			\eIf{animation is not running}{
				\breadthFirstSearch{treeData}\\
				remove old and inactive nodes
				}{
				wait until all animations are finished
			}
		}
    }
  \caption{Inkrementálne vykreslovanie modelu rozhodovacieho stromu..}
\end{algorithm}


%%%% Results %%%%%
\chapter{Vyhodnotenie a experimenty}
Pre kvantifikovanie správneho fungovania nami navrhovanej metódy navrhujeme viaceré experimenty. Prvým z experimentov je vyhodnotenie integrácie časti spracovania prúdu dát s našou metódou. Pri tomto vyhodnocovaní sa budeme pozerať na výkonnostné metriky, ktoré hovoria o priepustnosti a výkone tejto časti aplikácie. Zaujíma nás hlavne odolnosť voči chýbam, spracovanie v reálnom čase a zaťaženie procesoru a pamäte v závislosti na objeme dát.
\par
Vyhodnotenie samotnej metódy pre klasifikáciu prúdu dát nás zaujímajú bežné metriky používané pri vyhodnocovaní modelov strojové učenie, ako napríklad presnosť (angl. precision) a pokrytie (angl. recall). Keďže ide o klasifikovanie prúdu dát, zameriavame sa tiež na nasledujúce metriky:
\begin{itemize}
	\item \textit{Kappa štatistiky}, ktoré dobre vyjadrujú presnosť klasifikátora nestabilné prúdy dát.
	\item \textit{Najprv test-potom-trénovanie} (angl. Test-Then-Train alebo Prequential) je metrika používaná pre meranie výkonnosti klasifikátorov, ktoré sa vyvíjajú v čase.
\end{itemize}
Pri vyhodnocovaní klasifikačnej metódy sa tiež pozeráme na vhodnosť výberu rôznych algoritmov pre výber atribútov, detekcie zmien (angl. concept drift) a samotného klasifikačného algoritmu.
\par
Vysokú pozornosť venujeme vyhodnoteniu prezentovaných výsledkov používateľovi. Najprv robíme vyhodnotenie fungovania celej aplikácie ako celku s tromi expertami, ktorí majú detailné znalosti o fungovaní metód a algoritmov strojového učenia. Ďalej navrhujeme experiment vo forme používateľskej štúdie. Používateľská štúdia môže prebiehať v kontrolovanom, ale aj v "domácom" prostredí. Počas tejto štúdie budú účastnici vykonávať definované úlohy s cieľom zmerať a kvantifikovať ich výkonnosť a vôbec schopnosť splniť stanovené úlohy. Tieto úlohy môžu byť od jednoduchých ako odčítanie hodnotu z grafu, až po komplexné ako zistiť počet signifikantných zmien modelu a ich čas kedy nastali, či krátke slovná reprezentácia fungovania modelu. Prvé výsledky nami navrhovanej metódy sú na nasledujúcich dvoch grafoch. Zaiaľ sme nerealizovali experiment používateľskej štúdie.
\myFigure{images/exp-vis-concepts}{Na grafe je znázornený vývoj zmien v synteticky generovanom prúde dát. Každá zmena reprezentuje vznik alternujúceho podstromu.}{exp-vis-concepts}{0.5}{h!}\label{fig:exp-vis-concepts}
\myFigure{images/exp-vis-errors}{Graf zobrazuje vývoj chyby modelu a alternujúceho podstromu. Je možné pozorovať, že síce nastane moment, kedy je alternujúci podstrom kvalitnejší ako pôvodný, nikdy nesplní Hoeffdingovu mieru a teda nenahradí starý podstrom. Tento jav je spôsobený generovaním syntetického prúdu dát. Preto bude nutné v ďalších experimentoch použiť dáta reálneho sveta.}{exp-vis-errors}{0.5}{h!}\label{fig:exp-vis-errors}

V tejto kapitole navrhujeme metódu pre klasifikáciu prúdu dát. Metóda poskytuje iba jeden voliteľný parameter, ktorý musí nastaviť používateľ, hodnotu istoty $\delta$. Parameter reprezentuje istotu $1-\delta$, že bude výsledný model identický stým, ktorý by vznikol použitím tradičnej metódy. Cieľom metódy je, že používateľ nemusí mať znalosti o fungovaní klasifikačných algoritmov, ale je schopný použíť v praxi nami navrhovanú metódu. Naviac, výsledná model reprezentujeme vo forme vizualizácie, ktorý má pomôcť vysvetliť fungovanie modelu.

Kladieme si teda nasledujúce hypotézy:
\begin{hypothesis}{Naša metóda je schopná so stanovenou istotou klasifikovať prúdy dát a zároveň poskytuje výsledky v reálnom čase.}
\end{hypothesis}
\begin{hypothesis}{Metóda je ľahká na použitie a interpretované výsledky sú jednoduché na pochopenie pre doménového experta bez detailnej znalosti o fungovaní modelu.}
\end{hypothesis}
\begin{hypothesis}{Dokážeme zmysluplne a pochopiteľne, pre doménového experta, interpretovať blížiacu sa zmenu v uzle stromu a tiež zobraziť históriu zmien modelu.}
\end{hypothesis}





