\chapter{Klasifikácia rozhodovacími stromami}
\label{Klasifikácia rozhodovacími stromami}
Problém klasifikácie a jej definícia je podrobne opísaný v kapitole \ref{ulohy-klasifikacia}. V skratke, cieľom je nájsť funkciu $y = f(x)$, kde $y$ je skutočná trieda objektu/vzorky z prúdu dát a $x$ sú atribúty danej vzorky. Potom vieme pomocou funkcie $f(x)$ klasifikovať nové vzorky do triedy $y'$ s istou pravdepodobnosťou. My sa sústreďujeme na úlohu klasifikácie v doméne prúdu dát. V tejto kapitole opisujeme návrh spracovania prúdu dát, výber a aplikáciu metódy rozhodovacích stromov nad prúdom dát.

%%%%%%%%%%%%%%%%%%%%%%%%%
% Spracovanie prudu dat %
%%%%%%%%%%%%%%%%%%%%%%%%%
\section{Spracovanie prúdu dát}
\label{method-spracovanie-prudu-dat}

Spracovaniu prúdu dát venujeme samostatnú podkapitolu, pretože si zaslúži špeciálnu pozornosť a rozdielny prístup v porovnaní so spracovaním statickej kolekcie dát. Navrhovaná metóda je všeobecne použiteľná na problémy klasifikácie pre prúdy dát. Znamená to, že spracuje dáta v takmer reálnom čase, poskytne odpoveď, a teda aj vytvorený model a je schopná adaptácie na zmeny. Pre splnenie týchto požiadaviek je potrebné venovať samostatnú pozornosť spracovaniu prúdu dát, teda požadujeme, aby navrhovaná metóda spĺňala nasledujúce kritéria \citep{cimerman2015prudy}:
\begin{itemize}
	\item \textit{Odolnosť voči chybám} z pohľadu architektúry spracujúcej dáta. Chybné alebo chýbajúce dáta môžu mať kritický dopad na správne fungovanie a kvalitu klasifikačného modelu.
	\item \textit{Spracovanie v reálnom čase} je opäť dôležité pre správne fugnovanie výsledného modelu, pretože model je aktualizovaný a prispôsobovaný zmenám v dátach kontiunálne. Oneskorenie niektorých správ, napríklad o 24 hodín, čo je bežná prax pri ETL\footnote{ETL je proces, či architektonický vzor prenosu dát medzi viacerými častami databázových systémov  a aplikáciami, tento vzor je často používaný pre dátové sklady, skratka znamená Extrahuj, Transformuj a Načítaj (angl. Extract, Transform, Load)} procesoch, by mohlo mať nežiadúce následky vo forme skresleného modelu.
	\item \textit{Horizontálna škálovateľnosť} komponentu, ktorý spracuje prúd dát. Táto vlastnosť podporuje splnenie predchádzajúcich požiadaviek. Pod horizontálnou škálovateľnosťou chápeme to, že je možné zvýšiť výkonnosť celého systému pridaním fyzického uzla bez akýchkoľvek výpadkov. Táto požiadavka implikuje podmienku distribuovanej povahy riešenia.
\end{itemize}
%TODO: vyjadrit sa v zhodnoteni k tymto poziadavkam

S cieľom splniť tieto požiadavky navrhujeme použiť nasledujúce programovacie rámce a systémy:
\begin{itemize}
	\item \textit{Storm}\footnote{http://storm.apache.org/} je programovací rámec vytvorený pre spracovanie dát v reálnom čase. Storm poskytuje možnosti škálovateľnej architektúry, ktorá je naviac odolná voči chybám na úrovni kvality dát. Programovanie nad týmto rámcom je možné v každom programocom jazyku, ktorý je možné skompilovať do Java bajtkódu a vykonávať v JVM\footnote{Virtuálny stroj Java (angl. Java virtual machine)}. Storm poskytuje dovoľuje aplikovať akýkoľvek programovací vzor alebo model, ktorý je možné vyjadriť acyklickým orientovaným grafom zostrojeným z tzv. prameňov a skrutiek.
	\item \textit{Kafka}\footnote{https://kafka.apache.org/} je distribuovaná platforma pre spracovanie prúdov dát. Je vhodná na budovanie apliikácií, ktoré potrebujú spracovať zdroje dát v reálnom čase a vymieňať tieto dáta medzi aplikáciami. Poskytuje možnosť publikovať (angl. publish) a predplatiť (angl. subscribe) prúdy dát. Kafka je postavená na modeli fronty správ, pričom si tieto správy udržiava v pamäti a na disk ich replikuje pre prípad zlyhania. Kafka poskytuje distribuované a paralelné spracovanie dát, čo robí tento nástroj vhodný pre aplikácie reálneho sveta.
\end{itemize}

Nasledujúci obrázok schematicky popisuje architektúru spracovania prúdu dát potrebnú pre správne fungovanie metódy klasifikácie prúdu dát s použitím rozhodovacích stromov.
\myFigure{images/architecture}{Architektúra potrebná pre klasifikáciu prúdu dát v takmer reálnom čase. Architektúra pozostáva z troch úrovní. V časti predspracovania dát sú dáta zbierané zo zdroja prúdu dát a transformované do potrebnej podoby vhodnej pre ďalší krok. V kroku učenia modelu pre klasifikáciu prúdu dát je vytvorený klasifikačný model. Posledný krok obsahuje webovú službu, ktorá poskytuje Web API pre dotazovanie modelu. V tomto kroku tiež prezentujeme výsledky modelu v podobe vizualizácie používateľovi. Kafka je použitá na prenos správ medzi jednotlivými časťami aplikácie, správy sú rozdelené do rôznych tém podľa typu správy.}{architecture}{0.45}{h!}\label{fig:architecture}
\par
Web API rozhranie implementujeme ako asynchrónny web server. Volania Web API servera sú popísané v tabuľke \ref{tab-web-api}. Hlavným cieľom Web API rozhrania je možnosť klienta požiadať o posielanie aktualizácií modelu. Tieto aktualizácie sú odovzdávané v jednosmernej prevádzke vo forme Server-Sent Events (SSE), kde klient počúva a server môže posielať aktualizácie. Pri tejto forme komunikácie nie je potrebné pri každej správe otvárať nové TCP spojenie, čo je vhodné práve pre prúdové aplikácie. Pre každého nového klienta server vytvára nového Kafka konzumera. Nový konzumer je vytváraný vždy, pretože každý klient môže mať rôznu rýchlosť spracovania správ, a teda nastavený iné posunutie (angl. offset) Kafka konzumera.

\begin{table}[!htp]
\centering
\begin{tabular}{|r|l|}
\hline
\textbf{Cesta API volania} & \textbf{Popis} \\ \hline
/status/ & Vráti status web servera \\ \hline
/tree/ & Otvorí SSE spojenie a začne posielať aktualizácie \\ \hline
/node/\textit{node id}/ & Vráti informácie o danom uzle stromu \\ \hline
\end{tabular}
\caption{Popis Web API rozhrania webservera.}
\label{tab-web-api}
\end{table}

\section{Metóda klasifikácie prúdu dát}
\label{method-klasifikacia-prudu-dat}
Cieľom metódy je klasifikácia prúdu dát, pričom vytvorený model je pripravený na použitie takmer okamžite po prečítaní prvých vzoriek dát. Model sa tiež prispôsobuje zmenám a do istej miery sezónnym efektom v dátach. Základ metódy pre klasifikáciu sme zvolili state-of-the-art algoritmus rozhodovacích stromov, ktorý používa Hoeffdingovu mieru \citep{domingos2000mining, gaber2005mining, krempl2014open}. Hoeffdingova miera je použitá na rozhodnutie, či bol prečítaný dostatočný počet vzoriek na to, aby sa mohol list stromu zmeniť na rozhodovací uzol. Táto miera zabezpečuje to, že sa výsledný model asymptoticky blíži svojou kvalitou k tomu, ktorý by vznikol podobnou metódou pre statické dáta. Hoeffdingova miera je definovaná používateľom parametrom spoľahlivosti $\delta$, kde spoľahlivosť je $(1-\delta) \in <0,1>$. Metrika kvality vzorky $G$ môže byť ľubovoľná, napríklad informačný zisk (angl. information gain).
\par
Metóda potrebuje na trénovanie označkované numerické alebo kategorické dáta, musia byť vo forme n-tíc $(x_1, x_2, ..., x_n | y)$ kde $x$ sú atribúty vzorky a $y$ skutočná trieda vzorky. Nami vybraný algoritmus potrebuje len minimum parametrov potrebných nastaviť pre správne fungovanie. Jedným z nich je minimálny počet spracovaných vzoriek pred vytvorením prvého modelu. Týmto je možné minimalizovať prvotnú nepresnosť počiatočného modelu. Výsledný model je jednoduchý na reprezentáciu vďaka možnosti jeho intuitívnej interpretácie rozhodovacím stromom. Rozhodovací strom pozostáva z rozdeľovacích uzlov (angl. split node), tiež niekedy nazývané testovacie uzly, a listov (angl. leaf). V rozhodovacích uzloch sa vykonáva testovanie vzorky a jej posunutie do jednej z nasledujúcich vetiev alebo listu stromu. Ak vzorka narazí na list znamená to, že bola klasifikovaná do istej triedy, ktorú opisuje daný list. Takto vytvorený model je použiteľný na klasifikáciu v rôznych aplikáciách.
\par
Problémom rozhodovacích stromov je najmä ich šírka. Klasifikátory používajúce modely a algoritmy rozhodovacích stromov môžu byť podľa dát priveľmi široké. Tento problém môže mať za následok preučenie (angl. overfitting) modelu, ktorý bude vedieť klasifikovať veľmi dobre trénovacie dáta, resp. dáta zo začiatku prúdu, ale na nových dátach bude veľmi nepresný. Tento problém nastáva najmä pri spojitých číselných atribútoch a ich nerovnomernej distribúcii. Existuje niekoľko známych spôsobov ako sa s týmto nežiadúcim javom vysporiadať, jedným z nich je pre-prerezávanie (angl. pre-pruning) stromu. Tento spôsob aplikujeme aj v našej metóde priadním nulového atribútu $X_0$ do každého uzla, ktorý spočíva v nerozdeľovaní daného uzla. Takže uzol sa stane rozhodovacím iba vtedy, ak je metrika $G$ so spoľahlivosťou $1-\delta$ lepšia, ako keby sa uzol nezmenil na rozhodovací.
\par
Metóda sa musí vysporiadať so zmenami v dátach, pretože zmeny v dátach sú prítomné v takmer všetkých prúdoch reálneho sveta. Zmeny môžu mať rôzny charakter, napríklad náhly, kedy zmena nastane nečakane alebo postupný, kedy sa zmena deje dlhú dobu a pomaly. Výsledný model musí pre udržanie svojej presnosti zohľadniť tieto zmeny. Metóda používa algoritmus \textit{ADWIN} z anglického Adaptive Windowing \citep{Hutchison2009}. Tento algoritmus nepožaduje žiadne nastavenia parametrov používateľom ako napríklad veľkosť posuvného okna. Jediným parametrom je hodnota istoty $\delta$ s akou bude algoritmus detekovať zmeny v prúde dát. Myšlienka ADWIN spočíva v tom, že nenenastala žiadna zmena v priemernej hodnote vybranej metriky v okne. Ak je detekovaná zmena, v uzle začne narastať alternatívny podstrom. Tento podstrom musí spracovať definovaný minimálny počet vzoriek. Potom, ak je kvalita alternatívneho podstromu vyššia ako kvalita pôvodného podstromu, z ktorého začal narastať, pôvodný podstrom je nahradený alternatívnym podstromom. Naraz môže existovať niekoľko alternatívnych podstromov, pričom môže nastať situácia, kedy ani jeden z nich nebude mať vyššiu kvalitu a nesplní Hoeffdingovu mieru preto, aby nahradil pôvodný podstrom. Výsledok tohto algoritmu chceme detailne prezentovať používateľovi vo forme vizualizácie, ktorá je bližšie popísaná v nasledujúcej podkapitole. 
\par
Táto metóda potrebuje pamäť úmernú $O(ndvc)$ kde $n$ je počet uzlov stromu spolu s alternatívnymi stromami, $d$ je počet atribútov, $v$ je maximálny počet hodnôt na atribút a $c$ je počet tried. Znamená to teda, že pamäťová náročnosť algoritmu je závislá od štatistík, ktoré si strom udržiava, a úplne nezávislá od počtu spracovaných vzoriek z prúdu dát. Časová zložitosť spracovania jednej vzorky je $O(ldcv * log(w))$, kde $l$ je maximálna hĺbka stromu a $w$ je šírka ADWIN okna. Nakoľko algoritmus ADWIN používa variant techniky exponenciálnych histogramov s cieľom kompresie okna $w$, nemusí si celé okno explicitne udržiavať \citep{datar2002maintaining}. Vďaka tomu je časová a pamäťová náročnosť prechodu cez takéto okno o veľkosti $w$ len $O(log(w))$.
\par
Metódu implementujeme ako rozšírenie nad API, ktoré poskytuje nástroj Massive Online Analysis (MOA). Diagram tried implementácie je popísaný v prílohe XX.%TODO priloha moa impl
  Metódu by bolo možné jednoducho počítať distribuovane napríklad pomocu použitia rámca Storm.

%%%%%%%%%%%%%%%%%%%%%%%%%%%%%%%%%%%%%%%
% Prezentacia vysledkov pouzivatelovi %
%%%%%%%%%%%%%%%%%%%%%%%%%%%%%%%%%%%%%%%

\chapter{Vizualizácia modelu rozhodovacích stromov}
\label{my-method-prezentacia-vysledkov}
Vizualizáciou, ktorú navrhujeme adresujeme identifikované problémy existujúcich riešení v kapitole \ref{Existujúce vizualizačné a analytické nástroje}. Hlavným nedostatkom adresovaným nedostatkom je online vizualizácia modelu s dôrazom na zobrazenie zmeny modelu.
\par
V situácii, keď potrebuje doménovy expert vytvoriť klasifikačný model s použitím dát reálneho sveta, je často potrebná najprv detailná znalosť dát, ktorú doménový expert, predpokladáme má.  Následne preto, aby vedel vytvoriť správny model, potrebuje mať detailné znalosti o fungovaní klasifikačných metód a algoritmov. Cieľom našej metódy je odbremeniť experta od nutnosti disponovať podrobnými znalosťami o fungovaní modelu a algoritmov. Zameriavame sa teda na prezentáciu dôležitých informácií potrebných pre správne pochopenie modelu a následné rozhodnutia. Výber atribútov a algoritmov, ktoré budú použité na trénovanie modelu, je bez nutnosti interakcie používateľa v zmysle nastavovania parametrov a výberu metódy. 
\par
Keďže náš predpoklad je, že používateľ nemusí mať detailné znalosti o fungovaní algoritmov a klasifikačných metód, je dôležité vysvetlenie výsledného modelu. Znamená to, že je dôležité, aby pre používateľa nebol vzniknutý model len čierna skrinka (angl. black-box), ktorá s nejakou úspešnosťou dokáže klasifikovať prúdy dát. Navrhujeme preto vizualizáciu výsledného modelu. Vizualizácia je vo forme rozhodovacieho stromu, ktorý je jednoduchý na pochopenie aj bez predchádzajúcich znalostí rozhodovacích stromov \citep{nguyen2015survey}. Sústreďujeme sa najmä na zobrazenie zmien (angl. concept drift) v dátach, ktoré sa odzrkadlia zmenou modelu.
\par
Hlavným prípadom použitia doménovým expertom je:
\begin{enumerate}
	\item Vytvorenie modelu rozhodovacieho stromu.
	\item Doménový expert používa vytvorený model na strategické rozhodovanie v podniku.
	\item Doménový expert siahne po vizualizácii ak sa výrazne zníži kvalita modelu alebo potrebuje lepšie pochopiť, čo viedlo k vzniku aktuálneho modelu.
	\item Vizualizáciu je možné pozastaviť, pretože model sa inak neustále mení a vyvíja v závislosti od frekvencie a distribúcie prúdiacich dát.
\end{enumerate}

\section{Návrh vizualizácie}
\label{navrh-vizualizacie}
Vizualizáciu navrhujeme implementovať ako webovú aplikáciu. Zameriavame sa pritom na to, aby vizualizácia bola online, a teda zobrazovala vždy aktuálny stav modelu. Zobrazenie online meniacej sa vizualizácie modelu je pri modeloch prúdu dát dôležité pre lepšie pochopenie modelu. Táto vizualizácia má slúžiť najmä na zobrazenie priebehu učenia modelu v ktoromkoľvek momente s dôrazom na zobrazenie zmien, ktoré model ovpyvnili. V našej práci sme sa rozhodli vizualizovať model rozhodovacích stromov. Tieto rozhodovacie stromy, známe aj ako Hoeffdingove stromy, používajú Hoeffdingovu mieru ako mieru istoty pre výber ideálneho atribútu rozdeľovacieho uzla. Táto metrika nám napovedá to, či a ako sa model zmení. Vo vizualizácii využívame túto metriku na zobrazenie kvality listov alebo rozhodovacích uzlov stromu. Pomocou animácie zobrazujeme zmeny modelu, takže to prirodzene predstavuje evolúciu stromu. Na obrázku \ref{fig:vis-navrh} je zobrazený záber z vizualizácie.
\myFigure{images/6_navrh}{Vizualizácia dvoch iterácií modelu rozhodovacieho stromu. Modrá šípka zobrazuje prechod z pôvodného stavu do nového. Každý nový uzol v tejto vizualizácií vznikol samostatne a bol zobrazený animáciou. Model na obrázku bol vytvorený na syntetickom datasete.}{vis-navrh}{0.34}{h!}\label{fig:vis-navrh}
\par
Hodnota Hoeffdingovej miery je použitá na zafarbenie listov a rozhodovacích uzlov stromu. Listy stromu sú zafarbované na škále bielej a čiernej farby. Čím je nižšia Hoeffdingova miera v danom liste, tým je farba listu tmavšia. V ideálnom prípade sa teda každý čierny list zmení na rozdeľovací uzol. Niekedy môže nastať prípad, kedy sa aj menej tmavý list rozdelí. Je to z dôvodu, že Hoeffdingova miera je porovnávaná voči rozdielu chybovosti s rozostupom pozorovaní. Ak je tento rozostup nastavený napríklad na 200 vzoriek a práve v tomto krátkom okne sa Hoeffdingova miera zmení natoľko, že sa list rozdelí, potom sa nestihne prefarbiť na čierno. Tento jav sme pozorovali zriedka na reálnych dátach a aj na syntetických dátach. Pri rozdeľovaní listu je pôvodný list aj s vetvou vyfarbený na červeno a postupne vymizne. Na obrázku \ref{fig:vis-hb-color} je zobrazený prechod rozdeľovania listu na rozdeľovací uzol. V prípade, že v strome nenastala zmena, strom sa neprekreslí a aktualizujú sa iba farby uzlov. Animácia zobrazujúca rozdeľovanie listu zámerne ponecháva zobrazený aj pôvodný list práve preto, aby mohol používateľ dobre pozorovať zmenu, ktorá nastala.
\myFigure{images/6_hb-color}{Na obrázku je pre jednoduchosť zobrazený strom po spracovaní prvých vzoriek, preto na začiatku nie je vytvorený žiadny rozhodovací uzol. Šípka zobrazuje rozdelenie listu na rozhodovací uzol.}{vis-hb-color}{0.34}{h!}\label{fig:vis-hb-color}
\par
Keď nastane zmena v dátach ovplyvňujúca model, začne v danom rozhodujúcom uzle narastať alternatívny podstrom. Tento alternatívny podstrom začne narastať v takom rozhodovacom uzle, ktorého kvalita sa v čase začala znižovať. Ako detektor zmeny používame algoritmus ADWIN implementovaný v nástroji MOA. Pre samotnú vizualizáciu je tento detektor skrytý. Vizualizácia, ktorá slúži len ako front-end pre používateľa je informovaná o zmene z prúdu dát modelu stromov. Uzol, z ktorého začal narastať alternatívny podstrom, je označený červenou farbou. Červená farba zobrazuje vysokú pravdepodobnosť, že bude podstrom od daného uzla vymenený za jeho alternatívny podstrom. Okrem červenej farby v rozhodovacích uzloch je zobrazené aj explicitné upozornenie, že nastala zmena. Na obrázku \ref{fig:vis-change} je zobrazený model stromu s nastávajúcou zmenou. Výmena pôvodného postromu za jeho alternatívny podstrom je zobrazená rovnakou animáciou ako pri rozdeľovaní listu na rozhodovací uzol.
\myFigure{images/6_change}{Vizualizácia, kde existujú tri alternatívne podstromy (nie sú zobrazené). Upozornenie hovorí len o jednej zmene, pretože predchádzajúce dve zmeny, ktoré viedli k vzniku alternatívnych podstromov nastali skôr, ale nestihli zmeniť model stromu. Niektoré vetvy stromu nie sú úplne dokreslené kvôli lepšej čitateľnosti obrázku. Nečitateľnosť niektorých rozhodovacích pravidiel je v reálnej webovej aplikácií vyriešená posunutím pravidla do popredia po nájdení myšou na pravidlo.}{vis-change}{0.34}{h!}\label{fig:vis-change}

\section{Implementácia vizualizácie}
\label{Implementácia vizualizácie}
Vizualizácia bola implementovaná ako webová aplikácia pomocou knižnice D3.js\footnote{https://d3js.org/} v jazyku JavaScript. Knižnica D3.js umožňuje naviazať dáta na objektový model dokumentu (angl. Document Object Model, DOM) a jednoducho aplikovať dátovo riadené transformácie na dokument. D3 nie je monolitický programovací rámec vytvorený s cieľom poskytnúť každú funkciu, ktorú by mohol programátor potrebovať. Naopak, D3 rieši problém manipulácie a zmien DOM na základe dát. Pomocou tejto knižnice je jednoduché vytvoriť animované vizualizácie s rôznymi prechodmi, ktoré budú riadené dátovým modelom. Hlavným dôvodom prečo sme sa rozhodli pre túto knižnicu je možnosť funkcionálneho programovania, dátovo riadená vizualizácia a abstrakcia od vytvávarania SVG grafiky.
\par
Webová aplikácia získava potrebné dáta vo formáte JSON z Web API implementované pomocou už spomenutej architektúry Server-Sent Events (SSE). Z pohľadu vizualizácie to znamená, že prijíma prúd dát bez nutnosti otvárať TCP, resp. HTTP spojenie vždy, keď je potrebné prečítať novú vzorku. Znamená to, že server je schopný posielať aktualizácie webovej aplikácii bez toho, aby o tieto aktualizácie musela žiadať samotná webová aplikácia, napríklad periodicky. Toto je veľkou výhodou, pretože otvárať spojenie pri každej novej vzorke predstavuje značné nepotrebné zaťaženie naviac. Okrem toho, webové prehliadače efektívne implementujú tento štandard, pričom je možné bezproblémovo prijímať prúdy, ktoré generujú dáta s vysokou frekvenciou. Ukážka otvorenia SSE a asynchrónne spracovanie správy:
\begin{lstlisting}[language=JavaScript]
let eventStream = new EventSource(API_STREAM);
eventStream.addEventListener('tree', function (response) {
    let responseData = JSON.parse(response.data);

        BFT(responseData, addTreeIncrement).then((complete)=>{
            if (_.isEqual(complete, "success")) {
                log.info('Tree successfully processed');
            }
        removeAndFadeOutOldNodes(responseData, root);
    });
});
\end{lstlisting}
Dôležitou časťou spracovania údajov z tohto prúdu dát je práve asynchrónne spracovanie. Je to preto, že vizualizácia nežiada o aktualizácie manuálne alebo periodicky z dôvodu šetrenia prostriedkov a zaťaženia výpočtového výkonu a sieťovej prevádazky. Aktualizácie sú posielané vo vysokej frekvencii serverom. Keďže vizualizácia obsahuje animácie trvajúce približne 500ms, nový model stromu je potrebné spracovať nezávisle od nových vzoriek z prúdu dát. Keby sme každú novú vzorku spracovali v čase jej príchodu, vizualizácia by sa stala nečitateľná a nepoužiteľná. Po otvorení SSE spojenia a prijatí prvej aktualizácie vo forme modelu rozhodovacieho stromu vo formáte JSON je model vykreslený. Po prijatí ďalšej aktualizácie je model vykreslený, ak boli ukončené všeky animácie z predchádzajúceho vykresľovania. Vykresľovanie modelu stromu prebieha prehľadávaním do šírky, kde každý uzol je vykreslený samostatne s vlastnou animáciou. Je to z dôvodu, aby bola celý animácia plynulá. Algoritmus vykresľovania stromu je popísaný pseudokódom v algoritme \ref{alg:visualization}. Výpočtová zložitosť pridania uzlu je $O(|V| + |H|)$, kde $|V|$ je počet uzlov a $|H|$ je počet hrán. Skutočná časová zložitosť je znásobená časom každej animácie pri pridaní uzla, pri aktualizácií uzla je to len niekoľko milisekúnd, pretože netreba prekresliť žiadnu časť SVG.
\par
Navrhnutá metóda pre vizualizáciu rozhodovacích stromov učiacich sa online, poskytuje online animovanú vizualizáciu procesu tvorby modelu. Dôraz je kladený na zobrazenie zmeny modelu. Túto zmenu zvýrazňujeme červeným zafarbením rozhodovacích uzlov, ktorých sa zmena v dátach dotkla. Pre správne zobrazenie vizualizácie sme čelili viacerým technickým výzvam, a to najmä správnemu vykresľovaniu modelu stromu pri prijímaní prúdu s vysokou frekvenciou aktualizácií. Pri prezeraní vizualizácie môže používateľ zastaviť vizualizáciu alebo opäť ju spustiť. Okrem toho nie je možné inak ovládať vizualizáciu, pretože bola navrhnutá s cieľom jednoduchej interpretácie používateľom bez potreby nadbytočných vstupov z jeho strany. Túto hypotézu overujeme kvalitatívnym experimentom s niekoľkými doménovými expertami.
\par
% ====== ALGORITMUS visualization ====== %
\begin{algorithm}[H]
	\label{alg:visualization}
	\SetAlgoLined
    \SetKwInOut{Input}{Input}
    \SetKwInOut{Output}{Output}
    \SetKwFunction{visualize}{visualize}
    \SetKwFunction{breadthFirstSearch}{breadthFirstSearch}
    \SetKwProg{vis}{function}{}{}
    \SetKwProg{bfs}{function}{}{}
    
    \bfs{\breadthFirstSearch{$tree data$}}{
    	\Input{parsed tree JSON data}
    	
    	\For{\textbf{each} $node$ \textbf{in} $tree data$} {
			\eIf{node doesn't exists}{
				add new $node$ to model \\
				\textbf{\textit{await}} transition of $node$ is finished
			}{
				\textbf{\textit{await}} update node information
			}
		}
		
		\Return \textit{Promise(success)}
    }

    \vis{\visualize{$stream$}}{
    	\Input{Opened SSE stream}
    	\Output{Decision tree model SVG in DOM}
    	
    	\While{message received}{
			treeData $\leftarrow$ parse message data\\
			\eIf{animation is not running}{
				\textbf{\textit{await}} \breadthFirstSearch{treeData}\\
				\textbf{\textit{on success}} remove old and inactive nodes
				}{
				wait until all animations are finished
			}
		}
    }
  \caption{Inkrementálne vykreslovanie modelu rozhodovacieho stromu}
\end{algorithm}


%%%% Results %%%%%
\chapter{Vyhodnotenie}
V našej práci sme navrhli a implementovali online inkrementálnu vizualizáciu modelu rozhodovacích stromov, ktoré sa učia online nad prúdom dát. Pomocou nástroja MOA sme aplikovali algoritmus Hoeffdingových stromov s detektorom zmien ADWIN. Overenie pozostávalo z dvoch častí. V prvej časti sme overovali samotnú metódu klasifikácie a jej vybrané metriky. Overali sme tým to, či je táto metóda vôbec použiteľná na klasifikáciu rôznych prúdov dát. V druhej časti sme overovali samotnú vizualizáciu. Pri tomto overovaní sme sa sústredili na kvalitatívne vyhodnotenie s vybranou vzorkou doménových expertov. Návrh a implementácia vizualizácie je detailne popísaný v kapitole \ref{my-method-prezentacia-vysledkov}.

%%%% Kvantitativne vyhodnotenie %%%%
\section{Kvantitatívne vyhodnotenie klasifikácie}
Pre overenie použiteľnosti nami vybranej a aplikovanej metódy klasifikácie sme sa rozhodli overiť výkon tejto metódy. Cieľom tohto vyhodnotenia je overenie použiteľnosti metódy v aplikáciách reálneho sveta a to, že vybraná metóda neslúži len ako podpora pre vizualizáciu. Pri evaluácií výkonu klasifikátora nad prúdmi dát, ktoré obsahujú zmeny, sú dôležité dve metriky: \textit{presnosť} (angl. accuracy) a \textit{schopnosť adaptácie} (angl. ability to adapt).
\par
Prvá metrika by mohla byť meraná jednoduchou mierou chybovosti, čo znamená počet nesprávne klasifikovaných vzoriek alebo komplement tejto metriky, počet správne klasifikovaných vzoriek. Táto metrika je často používaná pri meraní výkonnosti klasifikátorov prúdov dát. Problém tejto metriky nastane vtedy, ak je distribúcia tried v prúde nevyvážená. Potom môže počet správne klasifikovaných vzoriek jednej triedy úplne prevážiť druhú triedu. Pričom presnosť nad minoritnou triedou môže byť pokojne nízka. Riešením tohto problému by bola metrika ROC alebo výpočet obsahu pod touto krivkou, známeho tiež pod názvom AUC. Problémom tejto metriky je, že ju nie je možné počítať inkrementálne nad prúdom dát, aj keď bola navrhnutá varianta pre prúdy dát \citep{brzezinski2014prequential}. Napriek tomu, že je táto metrika vhodná pre nevyvážené zdroje dát, jej použíteľnosť pre prúdy dát obsahujúce zmeny stále ostáva otvorený problém \citep{brzezinski2014prequential}.
\par
Vyhodnotenie druhej metriky, schopnosti adaptácie na zmeny, si vyžaduje samostatné merania. Niektorí výskumníci navrhujú merať schopnosť adaptácie reakčným časom adaptácie klasifikátora. Problém tohto prístupu je nutnosť expertnej evaluácie reakčného času.
\par
Z kvantitatívneho vyhodnotenia nás najviac zaujíma celková presnosť klasifikátora s odhliadnutím na niektoré vyššie spomenuté problémy. Je to z dôvodu, že sa sústreďujeme hlavne na kvalitatívne vyhodnotenie vizualizácie. Túto presnosť počítame procedúrou testuj-potom-trénuj (angl. test-then-train), kde je pre každé nové pozorovanie najprv vygenerovaná predikcia a následne je použitá na trénovanie. Zameriavame sa teda na metriku presnosti (angl. accuracy) a presnejšie jej priebežné počítanie. Priebežné počítanie je vykonané tak, že je pre každú novú vzorku najprv vypočítaná predikcia a porovnaná so skutočnou hodnotou vzorky. Podľa výsledku je vzorka zaradená medzi správne klasifikované vzorky, podľa ktorých je vypočítaná výsledná presnosť.

\subsection{Testovacie dáta}
Pre overenie presnosti klasifikátora sme použili dve dátové kolekcie dát z reálneho sveta. Presnejšie ide o výváženú statickú kolekciu dát s názvom \textit{Airlines (Air)}, ktorá obsahuje letové informácie v USA. Dáta sú zozbierané medzi rokmi 1987 a 2008. Presný výskyt a charatker zmien nie je známy. Napriek tomu tvrdíme, že sa zmeny v dátach nachádzajú. Jedna zmena nastala minimálne po 11.9.2001 kedy sa zmenilo mnoho pravidiel leteckej verejnej prepravy v USA \citep{brzezinski2014prequential, krawczyk2015one}. Úlohou nad týmito dátami je predikovať, či bude let omeškaný. Druhým datasetom, ktorý pozostavájúcim len 7000 vzoriek obsahoval dáta telekomunikačného operátora a cieľom bolo klasifikovať, či používateľ prestane používať služby operátora. Testovací prúd dát bol simulovaný inkrementálnym čítaním testovacích kolekcií.
\par
Problémom overovania algoritmov strojového učenia v doméne prúdov dát je nedostatok verejne dostupných datasetov pre vyhodnotenie výkonnosti. Mimoriadny problém predstavuje snaha o vyhodnotenie algoritmov, ktoré sa adaptujú na zmeny. Preto používame nástroj MOA na generovanie prúdov dát, kde kontrolovane zavádzame rôzne zmeny do prúdu dát. Generovanie prúdov syntetických dát s nastaveniami, ktoré boli použité na evaluáciu podobných algoritmov nám umožňuje lepšiu kontrolu nad experimentom. Takto vieme nami aplikovanú metódu klasifikácie porovnať s inými riešeniami \citep{krawczyk2015one, brzezinski2014prequential}. Presný popis nastavenia týchto generátorov je v prílohe YY. Používame nasledujúce dva generátory \citep{bifet2010moa}:
%TODO priloha s nastavenim generatorov
\begin{itemize}
	\item \textit{Hyp}, Hyperplane generátor bol použítý v troch variantoch s cieľom generovať prúd dát s inkrementálnymi zmenami. Tri varianty rôzneho pomery rozdelenia tried v prúde boli použité, 1:1 $Hyp_1$, 1:10 $Hyp_10$, 1:100 $Hyp_100$.
	\item \textit{RBF} generátor bol použitý na prúdu dát s náhlymi zmenami.
	\item \textit{$SEA_{ND}$} generátor bol použitý na vytvorenie prúdu dát bez zmien.
\end{itemize}

\subsection{Výsledky kvantitatívnych experimentov}
Nami aplikovaná metóda Hoeffdingových stromov s použitím ADWIN algoritmu ako detektoru zien dosahuje porovnateľné výsledky ako iné algoritmy pre klasifikáciu meniacich sa prúdov dát. Nami použitú metódu označujeme v tabuľke \ref{tab-method-results} ako \textit{VFDT-ADWIN}. Algoritmus Online Accuracy Updated Ensemble (OAUE) \citep{brzezinski2014prequential} si udržuje množinu komponentov klasifikátora s cieľom predikovať triedu novej vzorky váhovaným agregovaním súboru predikcií.
\par
Najprv sme chceli overiť možnosť monitorovania Hoeffdingovej miery s cieľom ju následne použiť na zobrazenie zmien vo vizualizácií. Monitorovanie Hoeffdingovej miery sa ukázalo ako vhodný indikatívny atribút opisujúci zmeny v konkrétnych častiach stromu. Nasledujúce dva obrázky zobrazujú chybovosť podstromu a jeho vzniknutého alternatívneho podstromu. Prvý obrázok \ref{fig:hb_never_replaced} zobrazuje situáciu kedy tento alternatívny podstrom nenahradil pôvodný podstrom, pretože rozdiel chybovosti nesplnil Hoeffdingovu mieru. Na druhom obrázku \ref{fig:hb_replaced} je možné vidieť situáciu kedy alternatívny podstrom nahradí pôvodný postrom.
\par
Na obrázku \ref{fig:synthetic-7k} a \ref{fig:real_7k} je zobrazený priebeh presnosti počas online trénovania. Obrázky zobrazujú len prvých 7053 vzoriek kvôli konzistentnosti, pretože toľko obsahuje Telco dataset. V prílohe XY sú zobrazené 
%TODO: priloha s grafmi
zvyšné obrázky s priebehom celého učenia. V tabuľke \ref{tab-method-results} sú porovnané výsledky s metódou OAUE. Podľa výsledkov považujeme nami aplikovanú metódu výkonnostne, v zmysle metrík na meranie výkonu klasifikátorov, porovnateľnú s metódou OAUE, ktorá dosahuje porovnateľné alebo lepšie výsledky ako všetky známe metódy \citep{brzezinski2014prequential}.

\myFigure{images/hb_never_replaced.pdf}{Priebeh chybovosti podstromu a jeho alternatívneho podstromu, ktorý vznikol pri detekovaní zmeny, resp. zvýšeniu chybovosti podstromu. Alternatívny podstrom nenahradil pôvodný podstrom.}{fig:hb_never_replaced}{0.9}{h!}\label{fig:hb_never_replaced}
\myFigure{images/hb_replaced.pdf}{Aletrnatívny podstrom nahradil pôvodný podstrom v momente keď sa rozdiel chyby rovnal alebo bol menší ako Hoeffdingova miera.}{hb_replaced}{0.9}{h!}\label{fig:hb_replaced}

\myFigure{images/synthetic_7k.pdf}{Priebeh presnosti pre jednotlivé synteticky generované dáta, zobrazených je prvých 7053 pozorovaní.}{synthetic-7k}{0.9}{h!}\label{fig:synthetic-7k}

\myFigure{images/real_7k.pdf}{Priebeh presnosti pre datasety reálneho sveta. Zobrazených je experiment na prvých 7053 pozorovaniach, pretože Telco dataset obsahuje len 7053 záznamov.}{real_7k}{0.9}{h!}\label{fig:real_7k}

\begin{table}[!htp]
\centering
\begin{tabular}{| c || c | c |}
\hline
\textbf{Zdroj dát} & \textbf{OAUE} & \textbf{VFDT-ADWIN} \\ \hline

$SEA_{ND}$ & 0.89 & 0.89 \\ \hline
$Hyp_1$ & 0.88 & 0.86 \\ \hline
$Hyp_{10}$ & 0.91 & 0.90 \\ \hline
$Hyp_{100}$ & 0.94 & 0.94 \\ \hline
$RBF$ & 0.99 & 0.89 \\ \hline
$Air$ & 0.67 & 0.60 \\ \hline
$Telco$ & - & 0.77 \\ \hline

\end{tabular}
\caption{Výsledky kvantitatívnych experimentov a porovnanie voči podobnému algoritmu. Meraná metrika je presnosť (angl. accuracy) predikcie vypočítaná metódou testuj-potom-trénuj.}
\label{tab-method-results}
\end{table}


\clearpage
%%%% Kvalitativne vyhodnotenie %%%%
\section{Kvalitatívne vyhodnotenie vizualizácie}
Hlavným prínosom našej práce je navrhnutá a implementovaná vizualizácia navrhnutá s cieľom uľahčiť pochopenie tvorby modelu rozhodovacích stromov a vizualizácii zmien, ktoré model ovplyvnili. Jadrom vyhodnotenia je preto kvalitatívne vyhodnotenie vizualizácie. To realizujeme ako riadenú používateľskú štúdiu s niekoľkými vybranými doménovými expertami, osobami ktoré sú znalcami v danej oblasti expertízy, napríklad domény internetových obchodov. Z nášho pohľadu doménový expert ale nemusí detailne poznať algoritmy strojového učenia.

\subsection{Používateľská štúdia}
Cieľom používateľskej štúdie bolo overiť, či vizualizácia napomáha pochopeniu modelu a zmeny, ktorá model významne ovplyvnila. Týmto kvalitatívnym experimentom sme tiež chceli zistiť, či používatelia správne chápu význam zafarbovania uzlov stromu. Experimentu sa zúčastnilo osem participantov, každý dobrovoľne. Dáta, na ktorých bol model vytvorený pre tento experiment boli vytvorené syntetickým generátorom MOA, pričom atribúty a triedy pripomínali dáta návštevníkov internetového obchodu. Synteticky generované dáta sme sa rozhodli použiť z dôvodu lepšej kontroly nad zmenami v dátach a teda aj celého experimentu. Pre experiment sme predpripravili tri videá, ktoré obsahujú záznam z priebehu učenia modelu. Každé video reprezentuje fázu experimentu:
\begin{enumerate}
	\item \textit{Fáza 0} je iniciálna fáza, v ktorej účastníka oboznamujeme s priebehom experimentu.
	\item \textit{Fáza 1} pozostáva z počiatočného učenia modelu. Toto video je najdlhšie a má necelých šesť minút. V tejto fáze pozorujme hlavne, či účastníci dokážu fázu správne slovne popísať a označiť vo videu čas od kedy sa začal model učiť.
	\item \textit{Fáza 2} obsahuje minútové video so stabilným modelom. Na začiatku videa nastanú ešte drobné zmeny v podobe rozdeľovania niektorých listov. Pozorujeme schopnosť popísať fázu a označiť moment od kedy je model v stabilnom stave.
	\item \textit{Fáza 3} zobrazuje učenie modelu s výraznou zmenou. Zmena je najprv detekovaná a vizualizovaná červeným zafarbením daných rozhodovacích uzlov, neskôr je uzol nahradný alternatívnym podstromom, čo predstavuje výraznú zmenu modelu.
\end{enumerate}
Každý participant spĺňal definíciu doménového experta. Všetci pracujú s dátami na dennej báze, vytvárajú základné analýzy ako napríklad lieviková analýza, či kontingenčné tabuľky a znalosti získané z týchto analýz používajú na strategické rozhodovania v podnikoch. Nikto z participantov nemal detailné znalosti o fungovaní algoritmov strojového učenia a nikdy podobné modely nevytvárali. Traja z ôsmich participantov podobný model použili v minulosti pre strategické rozhodnutia v podniku.
\par
Experiment prebiehal v známom prostredí a kontrolovanom. Na začiatku experimentu sme účastníkov oboznámili s jeho cieľom a snažili sme sa ich viesť k tomu, aby nám celý priebeh komentovali. Všetky ich komentáre sme si zapisovali, čo nám umožnilo získať náhľad na ich názor ešte predtým, ako videli a odpovedali na otázky, ktorými mohli byť ich odpovede čiastočne ovplyvnené. Počas experimentu sme účastníkom neodpovedali na žiadne otázky týkajúce sa samotnej vizualizácie alebo jej fungovania. Sumarizované výsledky používateľskej štúdie a sumár zápiskov z experimentu sú v nasledujúcej podkapitole. Odpovede v pôvodnom znení sú v prílohe X2. Znenie otázok používateľskej štúdie bolo nasledovné je v prílohe XY.
\par
Vyhodnocovacím kritériom používateľskej štúdie je správne označenie času záujmu vo videu podľa otázok a správne interpretovanie jednotlivých fáz. Správnu interpretáciu overujeme dvomi spôsobmi. Zápiskami ktoré vznikli z komentárov účastníkov počas experimentu pred odpovedaním na otázky. A druhým spôsobom je vyhodnotenie otázok, ktoré boli cielené na opis jednotlivých fáz experimentu, resp. vizualizácie.
%TODO priloha vysledky pouz. studie
%TODO priloha so znenim otazok


\subsection{Výsledky používateľskej štúdie}
Piati z ôsmych účatníkov mali menej ako dva roky skúseností v danom odbore, zvyšní traja mali 2-5 rokov skúseností v odbore. Účastník, ktorý použil v minulosti rozhodovacie stromy na strategické rozhodnutie mal práve 2-5 rokov skúseností. Napriek použitiu modelu v minulosti daný model nevytvoril.
\par
V prvej fáze siedmi účastníci správne vyjadrili slovný popis. Jeden účastník nevedel nijak popísať túto fázu. Popis bielej až čiernej farby listov bol správny pri štyroch účastníkoch, ktorý farbe priradili štatistickú významnosť. Ďalší dvaja opisovali farbu ako početnosť pozorovaní v danom liste alebo priradili bielej význam \textit{správnosť} a čiernej naopak \textit{nesprávnosť}. Tieto tvrdenia sa dajú považovať za čiastočne správne. Poslední dvaja účastníci nevedeli popísať farbu. Čas, od kedy sa model učí, označili všetci účastníci správne. Jeden z nich označil za začiatok učenia až desiatu sekundu videa, čo považujeme pri viac ako päť minútovom videu za správnu odpoveď.
\par
Druhá fáza zobrazovala model v stabilnom stave. Štyria z ôsmich účastníkov popísali správne fázu modelu. Znenie odpovedí sa síce líšilo. Traja odpovedali čiastočne správne odpoveďami ako napríklad: \textit{učiaca fáza} alebo \textit{testovanie predpokladov na výsledkoch}. Jeden účastník uviedol nesprávnu odpoveď, že sa model nachádza v prvej fáze predtým než sa začne prvýkrát učiť. Opäť sme sa pýtali na význam bielej až čiernej farby v listoch uzlov, odpovede boli rovnaké ako v prvej fáze. Model na videu sa dostal do stabilného stavu v 27 sekunde, prvých 26 sekúnd sa ešte rozdeľujú niektoré listy. Šesť účastníkov správne identifikovalo čas od kedy je model v stabilnom stave. Jeden účastník označil model za stabilný od začiatku videa a rovnako nesprávne popísal túto fázu.
\par
V tretej fáze sme overovali schopnosť pozorovať zmenu modelu. Popis tejto fázy bol najviac problematický. Len traja účastníci dokázali správne, tak ako sme očákavali, popísať túto fázu zmeny modelu. Jeden účastník túto fázu nazval ako \textit{fáza reštartu}, čo sa do istej miery dá považovať za správnu odpoveď. Zvyšní účastníci nazvali túto fázu ako učenie modelu. Okrem jedného účastníka všetci správne pripísali význam červenej farby rozhodovacích uzlov. Označenie času, kedy nastala zmena a momentu, kedy sa výrazne zmenil model rozhodovacieho stromu, označili siedmi účastníci správne. Jeden účastník nesprávne označil čas zmeny, pretože si nevšimol upozornenie o zmene, ktoré mal prekryté inou aplikácou zatiaľ čo červenú farbu uzla ihneď nepriradil zmene v dátach. Moment zmeny modelu tiež označil nesprávne a to v čase, kedy sa rozhodovacie uzly zafarbili červenou farbou. Na obrázku \ref{fig:graph-times} je možné vidieť označenie sledovaných časov vo všetkých fázach každým účastníkom experimentu.
\myFigure{images/graph-times.pdf}{Graf zobrazuje označený čas vo videu pre jednotlivé body záujmu. Vodorovné čierne čiary zobrazujú správnu odpoveď.}{graph-times}{0.45}{h!}\label{fig:graph-times}
Okrem otázok, na ktoré účastníci priamo odpovedali, sme sa snažili s nimi komunikovať a viesť ich k tomu, aby komentovali svoje myšlienky. Zo zápiskov vytvárame niekoľko tvrdení.
\par
Z počiatku boli všetci účastníci zmätení zmenou farby listu. Mnohí farbe najprv pripisovali počet pozorovaní v liste stromu, čo je čiastočne pravda. Približne od polovice experimentu začali všetci začali hovoriť o významnosti alebo dokonca o štatistickej významnosti uzla. Okrem jedného účastníka, všetci správne popísali význam farby uzlov, aj keď to neskôr v dotazníku nevedeli správne vyjadriť alebo si neboli istí a napísali nerozhodnú odpoveď. Niektorí účastníci z počiatku priraďovali farbu triede, ktorú klasifikuje list. Až neskôr správne pochopili, že farba nevyjadruje príslušnosť do danej triedy.
\par
Fáza, kedy bol model v stabilnom stave, bola prekvapivo najviac mätúca. Najprv účastníci popisovali túto fázu ako stav, kedy sa model účí, pričom si na chvíľu prestali byť istí farbami listov, pretože sa prefarbovali aj bez toho, aby sa ihneď rozdelili. Viacerí účastníci boli zmätení najmä od polovice vizualizácie, kedy bol model v stabilnom stave. Bolo to z dôvodu, že sa nemenil a neboli si istí, či sa niečo deje alebo experiment skončil. Viacerí sa preto pozreli na časomieru videa, či nezastalo.
\par
V tretej fáze správne popisovali významnosť červenej farby uzlov a výmenu pôvodného podstromu za alternatívny. Viacerí popisujú červenú farbu ako nestabilný alebo nekvalitný strom, či rozhodovacie pravidlo daného rozhodovacieho uzla. Jeden účastník najprv vyjadril nepochopenie k výmene stromu, ale neskôr túto výmenu správne popísal, aj keď do dotazníka napísal odpoveď \textit{neviem}. Ďalší účastník si nebol z počiatku istý významom farieb. Po výmene alternatívneho podstromu správne popísal význam červenej farby a výmenu podstromu ako nestabilitu modelu. Viacerí účastníci si nevšimli upozornenia o zmene v pravom hornom rohu vizualizácie.
\par
Po ukončení experimentu sme kládli ešte otvorené otázky smerujúce k tomu, čo by účastníci zlepšili alebo im chýbalo vo vizualizácii. Štyria účastníci vyjadrili potrebu zobrazovať aj samotný prúd dát, ideálne aj so znázornením distribúcie tried. Tvrdia, že by im to pomohlo v lepšom pochopení zmien a momentu, keď je model v stabilnom stave. Viacerí poukazovali na nepovšimnuté upozornenia. Účastník, ktorý nesprávne popísal takmer všetky fázy experimentu, po experimente tvrdil, že z počiatku nerozumel inkrementálnej a animovanej vizualizácii. Tiež nechápal tomu, prečo sa uzly rozdeľujú, ale po experimente hovorí že je to kvôli hľadaniu lepšieho rozdeľovacieho pravidla. Jeden účastník vyjadril nedôveru v tradičné dávkové metódy rozhodovacích stromov, totiž neverí tomu, že sa dá predikovať budúcnosť z minulosti a väčšiu dôveru v ňom vzbudzoval práve model meniaci a adaptujúci sa na zmeny. Vo všeobecnosti každý účastník konštatoval, že takáto vizualizácia zvyšuje jeho dôveru v samotný model. Veľmi pozitívne hodnotili práve zobrazenie červenej farby a vizualizáciu zmeny, pretože to považovali za dôležitý moment v životnom cykle modelu. Rovnako kladne hodnotili samotnú schopnosť adaptácie na zmeny.







































