%!TEX root = ./main.tex

\chapter{Úvod}
%\label{introduction}
V súčasnosti pozorujeme zvýšený záujem o oblasť analýzy a dolovania dát. Vhodné použitie a výber metód pre analýzu dát prináša hodnotné výstupy a náhľady pre doménového experta, ktoré môžu byť použité pre strategické rozhodnutia v podnikoch. Najčastejší postup je aplikovanie metód ako lieviková analýza alebo rozhodovacie stromy nad statickou kolekciou dát. Príkladom lievikovej analýzy môže byť snaha zistiť v ktorej časti nákupného procesu na internetovom obchode odchádza najväčšie percento zákazníkov. V prípade rozhodovacích stromov nás môže zaujímať klasifikácia zákazníka, ktorá ukáže či zákazník opustí internetový obchod. Na základe týchto výsledkov je možné dodatočne predložiť zákazníkovi jedinečnú ponuku a vďaka tomu zvýšiť zisk spoločnosti. Tento prístup má niekoľko problémov, medzi ktoré patrí predovšetkým: všetky trénovacie dáta musia byť uložené v pamäti alebo na disku; spracovanie a výpočtová náročnosť; vysporiadanie sa s trendami a zmenami v dátach. Nutnosť najskôr zozbierať a následne uložiť rýchlovznikajúce dáta predstavuje rovnako veľký problém ako ich samotné následné spracovanie.
\par
Pod pojmom spracovanie dát v reálnom čase rozumieme spracovanie v takmer reálnom čase, tzv. jemné (angl. soft) spracovanie v reálnom čase. Jemné spracovanie v reálnom čase v porovnaní s silným (angl. hard) spracovaním nezaručuje spracovanie vzorky v stanovenom čase, pričom niektoré vzorky sa môžu omeškať alebo úplne vynechať \citep{stankovic1988real}. Presné limity ohraničujúce spracovanie v reálnom čase závisia od konkrétného problému. Niekedy to môže predstavovat rádovo stotiny sekundy inokedy môže ísť rádovo o sekundy. V tejto práci budeme pracovať s pojmom spracovanie v reálnom čase chápajúc ho ako jemné spracovanie v reálnom čase.
\par
Pri dolovaní v prúde dát čelíme niekoľkým výzvam: objem,  rýchlosť (frekvencia) a rozmanitosť. Veľký objem dát, ktoré vznikajú veľmi rýchlo, je potrebné spracovať v ohraničenom časovom intervale, často v reálnom čase. Keďže sa objem dát neustále zväčšuje, potenciálne narastá až donekonečna. Identifikovali sme niekoľko najviac zasiahnutých oblastí, ktoré sú zdrojmi týchto dát: počítačové siete, sociálne siete, webové stránky (sledovanie správania používateľa na stránke) a Internet Vecí (angl. Internet of Things). Na informácie generované z takýchto zdrojov sa často pozeráme ako na neohraničené a potenciálne nekonečné prúdy údajov.
\par
Spracovanie, analýza a dolovanie v týchto prúdoch je komplexná úloha. Pre aplikácie je kritické spracovať údaje s nízkou odozvou, pretože riešenie musí byť presné, škálovateľné a odolné voči chybám. Nakoľko sú prúdy neohraničené vo veľkosti a potenciálne nekonečné, môžeme spracovať len ohraničený interval prúdu. V takomto prípade je potrebné dáta spracovať v čase ich vzniku. Tradičné metódy a princípy pre spracovanie statickej kolekcie údajov nie sú postačujúce na takéto prípady \citep{krempl2014open, han2011data}. Za tradičné metódy považujeme také, ktoré potrebujú najprv všetky dáta zozbierať, a potom nad vytvorenou dátovou kolekciou aplikujú metódy dolovania dát. Programovacie paradigmy ako MapReduce, na ktorých sú založené programovacie rámce ako napríklad Apache Spark \footnote{http://spark.apache.org/}, umožnili distribuované spracovanie veľkých objemov dát v akceptovateľnom čase. Problémom stále ostáva neustály nárast objemu dát, ktorý nie je ohraničený, a spracovanie v reálnom čase. Adaptácia na zmeny tiež často nie je zohľadená v týchto metódach. Zmeny môžu byť rôzneho charakteru, napríklad náhle alebo postupné, či opakované.
\par
Cieľom našej práce je analýza súčasných metód pre spracovanie a analýzu prúdu údajov a ich následné aplikovanie vybranej metódy v zvolenej doméne. Pre metódu analýzy prúdu údajov sme si zvolili metódu rozhodovacích stromov v úlohe klasifikácie. Kladieme dôraz na spracovanie v reálnom čase, ktoré je dôležité pri spracovaní prúdov dát. Veľkú pozornosť pritom mierime na schopnosť adaptácie metódy na zmeny v dátach. Výsledný model zobrazujeme vizualizáciou s cieľom uľahčiť pochopenie a interpretáciu výsledkov. Keďže neexistujú práce, ktoré sa priamo venujú a vyhodnocujú vizualizácie modelov nad prúdmi dát, rozhodli sme sa aplikovať vizualizáciu pre algoritmus rozhodovacích stromov učiacich sa v reálnom čase. Súčasťou práce je tiež návrh architektúry a prototyp implementácie potrebnej pre aplikovanie spomenutej metódy spolu s vizualizáciou.
\par
V druhej kapitole práce sa venujeme detailnej analýze rôznych analytických úloh nad prúdmi dát. Spolu s každou úlohou analyzujeme tiež metódy a algoritmy, ktoré sú vhodné pre jej riešenie. Jednou z týchto úloh je napríklad klasifikácia, ktorej venujeme najväčšiu pozornosť. Tretia kapitola hovorí o probléme interpretácie a vysvetlení výsledkov používateľovi. V štvrtej kapitole analyzujeme existujúce analytické a vizualizačné nástroje. V piatej a šiestej kapitole navrhujeme a popisujeme implementáciu metódy rozhodovacích stromov, potrebnej architektúry a vizualizácie. V poslednej kapitole vyhodnocujeme kvantitatívne a kvalitatívne implementovanú metódu.










