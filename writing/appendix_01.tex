\label{app.01}
\appendix
\chapter*{Prílohy}
\addcontentsline{toc}{chapter}{Prílohy}
\renewcommand{\thesection}{\Alph{section}}


% ======== DOKUMENTACIA ======= %
\section{Plán na zimný semester 2016/2017}\label{plan-zima}
\begin{itemize}
	\item Rozšírenie analýzy o ďalšie metódy a celkovo zpresnenie, zprehľadnenie a orezanie analýzy.
	\item Dokončenie návhu vlastnej metódy.
	\item Implementácia metódy.
	\item Evaluácia metódy - toto je priamo súčasťou navrhovanej metódy.
	\item Príprava článku na nejakú konferenciu, napr. IIT.SRC.
	\item Prvý experiment s použitím Eye Tracker-a na evaluáciu vizualizácie výsledkov.
\end{itemize}

\section{Vyjadrenie k plneniu plánu zimného semestra}\label{plan-zima-vyjadrenie}
Jedným z cieľov zimného semestra bolo rozšírenie analýzy. Môžme objektívne tvrdiť, že sa nám úspešne podarilo splniť tento cieľ. Časť analýzy projektu bola značne prehĺbená a rozšírená.
\par
Ďalším cieľom bolo dokončenie návrhu vlastnej metódy. Postupne a iteratívne sme celý semester pracovali na splnení tohto cieľa. Dnes, na konci zimného semestra máme presnú predstavu o tom ako má nami navrhovaná metóda vyzerať. Toto tvrdenie podporuje aj stav kapitoly \ref{Klasifikácia prúdu dát použitím rozhodovacích stromov}.
\par
Navrhovaná metóda bola implementovaná ako prvý prototyp. Tento prototyp ešte z ďaleka nepredstavuje finálnu verziu implementácie. Avšak, pomohol nám realizovať prvé jednoduché experimenty a teda aj výsledky, ktoré sú opäť prezentované v práci. Podarilo sa nám tiež implementovať prototyp, ktorý má jednoducho interpretovať a vizualizovať výsledky metódy požívateľovi.
\par
Síce sme implementovali prvé prototypy, evaluácia metódy bola len veľmi jednoduchá a základná. Napríklad nepodarilo sa nám splniť jeden zo stanovených cieľov - prvý experiment s použitím EyeTracker-a alebo používateľská štúdia.

\section{Plán na letný semester 2016/2017}\label{plan-leto}
Tento plán popisuje náš plán ďalšieho vývoja diplomovej práce v nasledujúcom letnom semestri na týždennej granularite. Pričom predpokladáme, že semester má 12 týždňov.
\begin{itemize}
	\item \textit{1-2 týžden}: Príprava článku na študentskú vedeckú konferenciu IIT.SRC. V prvých dvoch týždňoch semestra sa chceme sústrediť na dokončenie vyhodnotenie experimentov. Tieto výsledky chceme prezentovať na IIT.SRC, preto bude tomuto potrebné venovať viac času.
	\item \textit{3 týždeň}: Vyhodnotenie prezentovaných výsledkov na IIT.SRC, určenie ďalšieho smeru a priestoru na zlepšenie aktuálneho stavu implementovanej metódy.
	\item \textit{4. týždeň}: Implementácia návrhov na zlepšenie metódy pre klasifikáciu a ich vyhodnotenie.
	\item \textit{5. týždeň}: Implementácia návrhov na zlepšenie metódy pre vizualizáciu a ich vyhodnotenie. 
	\item \textit{6. týždeň}: Vyhodnotenie kvality a výkonnosti implementovanej metódy - kvantitatívne vyhodnotenie stanovených metrík kvality.
	\item \textit{7. týždeň}: Návrh ďalších experimentov vo forme používateľskej a expertnej štúdie v použiteľnosti navrhovanej metódy.
	\item \textit{8. týždeň}: Používateľská štúdia a spísanie výsledkov kvantitatívných metrík kvality metódy z 5-6. týždňa.
	\item \textit{9. týždeň}: Vyhodnotenie používateľskej štúdie.
	\item \textit{10. týždeň}: Analýza priestoru na zlepšenie navrhovanej a implementovanej metódy podľa výsledkov používateľskej štúdie.
	\item \textit{11. týždeň}: Spísanie výsledkov používateľskej štúdie a finalizácia diplomovej práce, príprava na odovzdanie.
	\item \textit{12. týždeň}: Odovzdanie diplomovej práce.
\end{itemize} 

\section{Technická dokumentácia}\label{tech-doku}
\textit{Obsah elektronického média}
\begin{itemize}
	\item \textit{Thesis.pdf} obsahuje elektronickú verziu tejto práce.
	\item Adresár \textit{moa-hfdt-extension} obsahuje rozšírenie nástroja MOA a experimenty, ktoré boli doteraz vykonané.
	\item Adresár \textit{vis} obsahuje prototyp vizualizácie so vzorovými súbormi.
	\item Adresár \textit{prototype} obsahuje prototyp webového rozhrania pre vizualizáciu.
\end{itemize}

\textit{Návod na inštaláciu}
\begin{itemize}
	\item Pre spustenie vizualizácie je potrebné povoliť v prehliadači cross-origin zdroje. Následné stačí dvojklikom spustiť súbor \textit{tree.html}.
	\item Pre spustenie experimentov je potrebné nainštalovať IDE Intellij\footnote{https://jetbrains.com/} a importovať adresár \textit{moa-hfdt-extension} ako nový projekt. Importovaný projekt bude obsahovať všetky potrebné nastvenia pre spustenie.
\end{itemize}




