\chapter{Zhodnotenie a budúca práca}
\label{Zhodnotenie a budúca práca}

V tejto práci sa venujeme analýze prúdu údajov s použitím rôznych metód pre analýzu údajov. Detailne analyzujeme najčastejšie úlohy analýzu dát, ktoré vykonávajú doménový experti. Tieto úlohy sú napríklad zhlukovanie, či klasifikácia. Zameriavame sa pritom na úlohu klasifikácie spolu s vizualizáciou vzniknutého modelu. Táto vizualizácia sa online inkrementálne mení, cieľom je zobraziť na zmeny, ktoré ovplyvnili model.
\par
Aplikovaná metóda pre klasifikáciu prúdu dát je rozšírením známej metódy rozhodovaích stromov. Algoritmus používa Hoeffdingovu mieru pre výber vhodného atribútu s obmedzeným počtom prečítaných vzoriek. Táto vlastnosť je žiaduca pri konštruovaní modelu nad prúdom dát. Naviac, istota výberu najlepšieho atribútu pre rozhodnutie rastie spolu s počtu prečítaných vzoriek. Pozornosť venujeme tiež adaptácii na zmeny (angl. concept drift) v prúde dát. Aplikujeme algoritmus ADWIN s cieľom adaptívneho posuvného okna, ktoré zabezpečí adaptáciu modelu na rôzne typy zmien. Z kvantitatívnych experimentov možno pozorovať, že nami vybraná metóda klasifikácie dosahuje porovnateľné výsledky ako podobná metóda.
\par
Hlavným prínosom našej práce je vizualizácia metódy klasifikácie rozhodovacími stromami nad meniacim sa prúdom dát. Navrhli a implementovali sme vizualizáciu, ktorá sa mení inkrementálne online v čase tak ako sa mení model. Červenou farbou zobrazujeme šancu to vznik alternatívneho podstromu a zvýšenú šancu na nahradenie jeho pôvodcu. Týmto je vo vizualizácií zobrazená zmena v dátach. Nahradenie pôvodného podstromu jeho alternatívnym podstromom je vizualizované animáciou.
\par
Pre overenie použiteľnosti vizualizácie sme vykonali používateľskú štúdiu s ôsmimi doménovými expertami. Študovali sme, interpretáciu zafarbovania uzlov a animácií, ktoré zobrazli buď inkrementálne učenie modelu alebo zmenu zmenu v modeli. Na základe výsledkov štúdie môžme tvrdiť, že naša vizualizácia pomáha pri interpretovaní online meniaceho sa modelu rozhodovacích stromov nad prúdmi dát.
\par
Ďalšou prácou bude obohatenie vizualizácie o náhľad na samotný prúd dát, ktorý chýbal viacerím účastníkom experimentu a pomohol by im lepšie pochopiť niektoré fázy modelu. Pridanie tretej dimenzie by malo pridanú hodnotu v možnosti vizualizácie alternatívnych podstromov. Čo v 2D vizualizácií nebolo možné kvôli zachovaniu prehľadnosti. Ukladanie priebežnej histórie a spätný pohľad je tiež jeden zo smerov budúcej práce, ktorému by sme sa chceli venovať. Samotná metóda klasifikácie by sa v prípade použitia nad prúdmi reálneho sveta dala nasadiť do distribuovaného počítania napríklad pomocou rámca Storm, čím sa vyrieši škálovateľnosť metódy.